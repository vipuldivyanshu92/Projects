\documentclass[11pt,a4wide]{article}
\usepackage{a4wide}
\usepackage{graphicx}
\usepackage{float}
\begin{document}
\section{Team Organisation}
\subsection{Introduction}
The team name is ‘Team LEAD’ where LEAD stands for “Linux Embedded Automotive
Dashboard”. The team consists of three members:
\begin{itemize}
	\item Amal Kakaiya
	\item Simon Jouet
	\item Esiri Igbako
\end{itemize} 

An original fourth member, left the team and course due to other project commitments.

\subsection{Communication}
Weekly meetings of the team, and meeting with project supervisors, will track
the project progress against the team Gantt chart. These meetings will be used 
to assess deadlines, revise requirements, delegate tasks, and organise meetings 
with UGRacing (the client) should any clarification of requirements be needed.

\subsection{Development and Tracking}
Outside of formal meeting times, team members are expected to perform delegated 
tasks to deadlines and track progress through a ticketing system, Trac. This, 
along with the Subversion (SVN) versioning repository will ensure that the team 
has a well documented and transparent development process. Weekly time-sheets 
will also be filled in to match Trac tickets for further documentation. All 
team work, including code, documents, schematics, diagrams, data-sheets and 
sources are required to be added to the SVN repository throughout the 
development process. Not only is this a safe and secure means of keeping the 
data (providing the repository is backed up), it also allows all team members 
to work on up-to-date versions of all documents in the development process. 
Which allows concurrent development on code and documents.

\section{Risk Management Plan}
As part of the development process of the proposed system, the team has 
calculated the following potential risks, and probabilities for the risks 
becoming real, along with the forecasted consequences.

\subsection{Risk 1: Component Compatibility Issues}
Concerns have been raised about the fact that even though the components we 
have selected may conform to the same standards (for CAN), they might not work 
together as expected. Namely the Microchip and Freescale components.\\
\\
Probability: Medium - High\\
Impact: Medium\\
Contingency actions: Select different components, and work on 
finding/deveoping Linux drivers for them.

\subsection{Risk 2: Component failure}
This is mostly concerning the main COM Gumstix component. If this fails, then 
the entire project is compromised as the device is expensive and takes a while 
to deliver.\\
\\
Probability: Medium\\
Impact: High\\
Contingency actions: Multiple Gumstix (2 max) could be ordered just in case. 
However, this is expensive, so it will not be done. Other components can easily 
be ordered as they are fairly inexpensive and quick to obtain. 

\subsection{Risk 3: Team integration}
Poor inter-team communication (with teams developing other components) means 
that we would be may be to interpret data generated by other teams. Since we 
are collecting data from all teams, for display on our dashboard, standards 
must be agreed and adhered to in order for the project to succeed.\\
\\
Probability: Medium\\
Impact: High\\
Contingency actions: Agree standards with other teams and adapt code accordingly.

\subsection{Risk 4: Personnel loss}
In the event that a team member leaves the team and/or course, we must be able 
to prepare for the remaining workload.\\
\\
Probability: Low\\
Impact: High\\
Contingency actions: Re-distribution of workload. Potential case for extension 
of deadlines. Revised Gantt chart required.

\subsection{Risk 5: Misunderstood requirements}
Our understanding of UGRacings requirements for the dashboard, may be different 
from their understanding. \\
\\
Probability: Medium\\
Impact: Varying\\
To ensure that both sides have equal understanding, an iterative requirements 
gathering process can be used until an agreement is met.

\section{Conclusion}
The team has put a framework in place to allow for smooth development and 
delivery to the client. Risks have appropriate contingency plans, and the team 
should be able to deal with these forecasted risks should they become real. 
Unexpected risks will require further risk assessment and plan revision, 
depending on the severity of their impact.

\end{document}
